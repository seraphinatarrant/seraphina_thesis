%thesis.tex 
%Model LaTeX file for Ph.D. thesis at the 
%School of Mathematics, University of Edinburgh

\title{Fairness in Transfer Learning for Natural Language Processing}
\author{Seraphina Goldfarb-Tarrant}
\date{2023}

\documentclass[phd,ilcc,oneside,leftchapter,parskip]{infthesis}

\usepackage[T2A,T1]{fontenc}
\usepackage[utf8]{inputenc}

\usepackage[english]{babel}

\usepackage{natbib}

% \usepackage{biblatex}
% \addbibresource{biblatex_bib.bib}

\usepackage{latexsym}
\usepackage{graphicx}
\usepackage{wrapfig}
\usepackage{times}
\usepackage{dirtytalk}



\usepackage[dvipsnames]{xcolor}

\usepackage{subcaption}
\usepackage{mwe}
\usepackage{booktabs} % Top and bottom rules for tables
\usepackage{svrsymbols}

\usepackage{tikz-dependency}
\usepackage{tikz-qtree}
\usepackage{paralist}
\usepackage{tikz}
\usetikzlibrary{shapes,arrows,positioning}
\usetikzlibrary{matrix}
\usepackage{etoolbox}
\usepackage{bm}
\usepackage{amssymb}
\usepackage{amsfonts}
\usepackage{amsmath}
\usepackage{csquotes}
\usepackage{verbatim}
% \usepackage{todonotes}
% \usepackage[utf8]{inputenc}
\usepackage{CJKutf8}

\usepackage{epigraph}
\setlength{\epigraphwidth}{0.7\textwidth} 

\usepackage[final]{pdfpages}

\usepackage{url}
\usepackage{hyperref}
\usepackage{cleveref}
\urlstyle{rm}
\definecolor{revcolor}{rgb}{0.81, 0.09, 0.13}
\definecolor{darkblue}{rgb}{0.2, 0.2, 0.6}


\usepackage[left, mathlines]{lineno}
\renewcommand\linenumberfont{\normalfont\bfseries\small\color{lightgray}}

\hypersetup{
    colorlinks,
    citecolor=darkblue,
    filecolor=black,
    linkcolor=darkblue,
    urlcolor=darkblue
}

% \usepackage{lsubfiles}

\newcommand{\sgtcomment}[1]{\textcolor{blue}{[ST: #1]}}

\newcommand{\E}{\mathbb{E}}


\abstract{Natural Language Processing (NLP) systems have come to permeate so many areas of daily life that it is difficult to live a day without having one or many experiences mediated by an NLP system. These systems bring with them many promises: more accessible information in more languages, real-time content moderation, more data-driven decision making, intuitive access to information via Question Answering and chat interfaces.
%and scale and efficiency at low costs. 
But there is a dark side to these promises, for the past decade of research has shown that NLP systems can contain social biases and deploying them can incur serious social costs. Each of these promises has been found to have unintended consequences: racially charged errors and rampant gender stereotyping in language translation, censorship of minority voices and dialects, Human Resource systems that discriminate based on demographic data, a proliferation of toxic generated text and misinformation, and many subtler issues. 

Yet despite these consequences, and the proliferation of bias research attempting to correct them, NLP systems have not improved very much. There are a few reasons for this. First, measuring bias is difficult; there are not standardised methods of measurement, and much research relies on one-off methods that are often insufficiently careful and thoroughly tested. Thus many works have contradictory results that cannot be reconciled, because of minor differences or assumptions in their metrics. Without thorough testing, these metrics can even mislead and give the illusion of progress. Second, much research adopts an overly simplistic view of the causes and mediators of bias in a system. NLP systems have multiple components and stages of training, and many works test fairness at only one stage. They do not study how different parts of the system interact, and how fairness changes during this process. So it is unclear whether these isolated results will hold in the full complex system. Here, we address both of these shortcomings. We conduct a detailed analysis of fairness metrics applied to upstream language models (models that will be used in a downstream task in transfer learning). We find that a) the most commonly used upstream fairness metric is not predictive of downstream fairness, such that it should not be used but that b) information theoretic probing is a good alternative to these existing fairness metrics, as we find it is both predictive of downstream bias and robust to different modelling choices.   %illuminating the inconsistencies in a number of common metrics, and making recommendations to improve their utility. 
We then use our findings to track how unfairness, having entered a system, persists and travels throughout it. We track how fairness issues travel between tasks (from language modelling to classification) in monolingual transfer learning, and between languages, in multilingual transfer learning. We find that multilingual transfer learning often exacerbates fairness problems and should be used with care, whereas monolingual transfer learning generally improves fairness. Finally, we track how fairness travels between source documents and retrieved answers to questions, in fact-based generative systems. Here we find that, though retrieval systems strongly represent demographic data such as gender, bias in retrieval question answering benchmarks does not come from the model representations, but from the queries or the corpora. We reach all of our findings only by looking at the entire transfer learning system as a whole, and we hope that this encourages other researchers to do the same. We hope that our results can guide future fairness research to be more consistent between works, better predictive of real world fairness outcomes, and better able to prevent unfairness from propagating between different parts of a system. 
}

\begin{document}

%% First, the preliminary pages
\begin{preliminary}

%% This creates the title page
\maketitle

\begin{laysummary}
\textbf{Natural Language Processing} describes any AI system that deals with human language. Today NLP systems are part of every person's daily life; visibly as phone voice assistants and automatic YouTube captioning, and invisibly, when that person applies for a job or when all the data they put online is analysed for content moderation, marketing, opinion polling, and more. \textbf{Transfer Learning} describes systems that are multi-part, which almost all systems are today, because they are built from one big language model, like ChatGPT, Llama, Mistral, or Cohere, which is later \textit{fine-tuned} to a particular task, such as customer service or resume processing. Fine-tuning just means that some data in that domain, usually expensive, often proprietary, is used after the model is trained in order to make the language model less general purpose and better at that type of data. This system described so far is Transfer Learning. In many of today's systems, the language model is then connected to a RAG (Retrieval Augmented Generation) component, where the model can search a database of documents: Wikipedia, confidential medical records, all the PDFs that a PhD student has downloaded over the course of their degree\footnote{Roughly 1227 papers, in my case.}. The premise of the research in this thesis is that you have to know how the parts of these systems interact, in order to judge whether a system is \textbf{fair}. This means the interaction between all of: the original big language model, the data you fine-tune on, anything else you make at the fine-tuning stage, the retrieval system, the corpus it retrieves from. A system that is \textbf{fair} is one that doesn't propagate social biases and stereotypes, and doesn't screw over minority groups in society. To be precise to the definition of fairness, it shouldn't screw over \textit{any} group, even straight white men, but the minority ones tend to be the ones we worry about since they tend to be most screwed over, with some exceptions. 

In this work, we discover a couple of things about how the parts of an NLP system interrelate, with regard to fairness. First, you cannot measure fairness of just the language model in isolation and know whether your model will be fair or unfair when it's deployed in an application later. You can get an indication as to the model's \textit{potential} for unfairness, but that is all. This idea of \textit{potential} can be understood by analogy to genetics---we can measure whether a person is more or less likely to develop cancer, but whether they do or not over the course of their life depends on whether they work in copper mines and what they eat, how much they exercise and what pollutants are in the air where they live. Some people with a high propensity may never develop it, and some people with no propensity at all still will if they live next to Chernobyl. With language models, we can measure how much potential there is for unfairness, but we can't know what this really means without the environmental factors: the fine-tuning data, the RAG system, and even the cultural context of deployment, which determines what demographic groups are considered minorities. We can make predictions about how likely the model will be to become unfair, if it is fine-tuned on skewed data that doesn't have positive examples of good resumes for people who aren't white, or doesn't have high quality RAG data for people who are not male (this is, incidentally, true of Wikipedia \citep{sun-peng-2021-men}). So we can improve, or mitigate this \textit{potential} in a language model. But in real world applications, where we need \textit{know} that people are not screwed over with some degree of certainty, we need to test the final system.

Most language models today are also \textbf{multilingual}. If you type something into ChatGPT or Cohere's model in English, it can retrieve information from documents in Turkish and summarise them for you in English. If you type something in Korean, it will respond in Korean. In this work, we discovered that data in one language can influence fairness behaviour in \textit{another} language. An NLP system that handles Japanese can become more sexist and racist when you add data from English, even though the data you're adding is not in Japanese, and even though the sexist and racist stereotypes in English data are not the same as those in native Japanese. 

Overall, this work shows that you cannot assume that the addition of new data---from fine-tuning, for other languages, from RAG---does not change the fairness properties of an original model. So fairness cannot just be the domain of the tech giants and gold-plated start-ups, of Google, Meta, Anthropic, Cohere, or whatever companies are producing the new hot model next year. Is has to be a collaboration between the people training the big models, the people deploying systems in the real world, and, ideally, the users. 

\end{laysummary}

% Acknowledgements
\begin{acknowledgements}
People say that PhDs are the most solitary time of your life. In some ways, I've found that to be true. But I very much did not do this alone. 

The inkling in my mind that I wanted to leave my career and go study NLP began roughly ten years ago, and I would not be here today without the herculean efforts of some, and the small graces of many. I want to thank first Somusa Ratanarak, who encouraged me to leave Google and pursue my passions when all my colleagues thought leaving a stable job was utterly insane. She's been with me every step of the way, with a lot of cleverness and care, and a little bit of \begin{CJK*}{UTF8}{ipxm} 愛の鞭 \end{CJK*}, 
and I am a much better person because of her. I want to thank my parents, all of them, Mom\#1, Mom\#2, Dad, Tyger, and Rachel, who also didn't think I was crazy. Thank you for seeing and celebrating my accomplishments, with flowers and lemons and jam, even when I didn't stop to appreciate them because I was busy running for the next goalpost. Thank you for being proud of me.

There were still others whose help I needed even to reach the start of my PhD. I want to thank Ozan Mindek, for giving me my first chance at NLP on a 20\% project. Then Emily Bender, for walking her talk and creating an MSc program that was \textit{truly} diverse in practice. She took me with my undergraduate in Ancient Greek, and my now friends Brian from the Navy and Lonny working on Yupiq preservation, along with the usual CS suspects. I aspire to be that conscientious in organisations that I lead, and I can see already how challenging it will be to live up to that. I also need to thank that same diverse cohort of Brian, Lonny, Catharine, Genevieve, Amandalynne, Chris, and Ben for spending Saturday mornings helping me practice interview for my PhD.
And I aspire also to be like two more women who helped me: Fei Xia, who can make absolutely any question you ask her interesting, no matter how silly it was, and who showed me the beauty and elegance of statistical machine learning. And Nanyun Peng, for taking on a green MSc student and teaching me to write papers and staying up late with me, and showing me that I absolutely loved doing research. 

Now we have reached the start of my PhD. Here I want to thank both my advisors. Björn, for taking me on partway through and bringing his energy and enthusiasm to keep me going, at a point in the pandemic when my own energy was flagging; our Meadows walks kept me going. Adam, for accepting me, trusting whatever direction I ran off in, and immediately being for me a mentor who cares deeply about both good science and about people. Adam has changed the way I do science and the way I see the world; I am proud to be in his intellectual lineage. 

So many other people made my PhD possible, by keeping me afloat along the way, in a period in which very many things went wrong in my life and in the world. I owe to them, and to those moments, both a piece of this research and of who I am. 

Late nights walking the City of London with Sameer Bansal, whose dark humour and enormous generosity lightens all loads. Hikes, bothies, wine and cheese nights with Kate McCurdy, who would always make me laugh and teach me something marvelous about humans or language, and who became a better friend and colleague than I could've ever wished for. Always-inadvisably-late nights by the fire with Henry Conklin, with whom I get my most interesting NLP inspiration and also with whom I can be my whole self, unfiltered. All my adventures in the beautiful wilds with Ida Szubert, who holds so much of my soul and my delight in life, and with whom it is so easy to share space, and somehow never a chore to work late. Pints at Sandy Bells with Ramon Sanabria, who sees the cheerful beauty in even quite shitty situations and shows me how to do this too. Dinners and walks with Yevgen Matusevych, helped me when I needed it and scolded me for not asking. Coffees with Tom Hosking, who has a knack for re-energising me about the whole field of NLP over coffee, just when I've been banging my head against a problem so long it's become tedious. Writing accountability meetings at the Burn with Tom Sherborne, who was instrumental in getting my nightmare multilingual work onto a page, and has been a companion since. Brewing the worst beer I have ever made with Sander Bijl de Vroe and drinking it anyway. Walks down the Innocent Railway with Siddarth Narayanaswamy, reminding me that I wished I'd learnt first principles better, but also that this wish is why I became a scientist in the first place. Whisky nights with Jasmijn Bastings, as well as our long chat on my 15-hour solo drive south from NAACL after I caught Covid-19 there. She was always so easygoing, and foolish/caring enough to come to all my panels and talks, without me asking. 

All of my direct collaborators also kept me going through a PhD that was much, much lonelier than I could have ever anticipated. Meetings with my collaborators Hadas Orgad and Yonatan Belinkov, which I would look forward to from the start of every week and would keep me energised until the end of it.  Adam and Sharon's Agora research group in ILCC provided feedback, structure, and scholarly companionship. Diego Marcheggiani and Roi Blanco helped me manage my first enormous multilingual project. Patrick Lewis and Pedro Rodriguez exemplified a combination of rigor and curiosity in their science, as well as conscience, that I admire and hope to live up to. 

I am thankful for the small graces (and some large graces) far beyond the research sphere also. To James and Deena Owers-Bardsley for my intro to cycling in Scotland and for the cocktail kits we left on each others' doorsteps when we couldn't see other people. To David Halliwell and Narma Gebruk, for celebrating my intended submission with me last May, and still sticking with me for the next nine months. To the entire Beltane family --- especially my performance groups Veles, Goblin Fire Arch, Goblin Bower, The Summer King, and Obsidian --- who gave me a haven away from academia which at times I desperately needed. To Alison Stewart, for taking me in practically as a family member after that one coffee in George Square. To Sam Roots and Ruari Cathmoir, for being my chosen family. To Ellen Mears, who has been my companion in doing the things that I want to do for \textit{me}, rather than spending all my time on my sometimes heavy responsibilities. To Guru Khalsa, for every one of those calls from the road. To Ivan Ivanov, who was accidentally stuck with me during the pandemic, and who I hope to get stuck with many more times in my life. To James Hartley, who is one of the few people in the world I feel I can lean on, and who has always provided me with so much love it keeps me warm from another hemisphere. To Ezra Baydur, for drunk-chat-chess and seeing the best in me. To Craig Innes, who has walked beside me and introduced me to a version of myself that I didn't know was there before, and that I am so glad to have gotten to know.  Who reliably has something uplifting to say about my skills, to prop me up, before I give a scary public appearance. To Tasuku Koda, Narumi Ota, and Makoto Takashima, for showing me \begin{CJK*}{UTF8}{ipxm} 本音 \end{CJK*} and keeping a place on the other side of the world that feels like home. To Alice in Slumberland, my Burning Man family, who I saw only once during my PhD but who taught me that I could express myself any way I wanted, and that I could even redefine the world in which I lived. I'm trying to do a little bit of that redefinition in here, with this work. 

\end{acknowledgements}


%% Next we need to have the declaration.
%\standarddeclaration
\begin{declaration}
   I declare that this thesis was composed by myself,
   that the work contained herein is my own 
   except where explicitly stated otherwise in the text,
   and that this work has not been submitted for any other degree or
   professional qualification except as specified.

   My specific contributions are as follows for each chapter:
   
   Chapter~\ref{chapter:intrinsic_bias_metrics}: I designed the research agenda: I envisioned the research question and wrote a research plan document with methods, goals, metrics, and a literature review. I recruited and supervised three MSc students who implemented pipelines for three different systems and did initial investigations into the correlation between intrinsic and extrinsic metrics for their MSc theses. I gathered these pipelines together, extended them, ran experiments, and wrote and presented the paper, with the help of Adam. 

   Chapter~\ref{chapter:gender_bias_probing}: I was second author on this paper, assisting the first author Hadas Orgad who proposed this extension of the work in Chapter~\ref{chapter:intrinsic_bias_metrics}. I implemented some of the metrics on intrinsic analysis of language model representations, implemented the additional extrinsic fairness metrics (which are now open-sourced), and co-wrote the paper with Hadas Orgad and Yonatan Belinkov.

   Chapters~\ref{chapter:multilingual_sentiment_analysis} and \ref{chapter:multilingual_sentiment_analysis_pt2}: I did this project almost entirely on my own save for the writing, which my supervisors Adam and Björn assisted with greatly. I designed the research question and outlined the project, developed and programmed the experiment framework, found training data, created evaluation data (with the help of native speakers of each language) and wrote up the results (with the help of my supervisors). I received regular weekly consultation from Diego Marcheggiani, Roi Blanco, and Lluis Marquez at Amazon Barcelona for the first of the two projects.

   Chapter~\ref{chapter:contrievers}: I designed the research question in collaboration with Patrick Lewis. I made the research project plan, chose datasets and interpretability methods, wrote all pipeline code and ran all experiments, and finally wrote up the findings into a paper, with help from Pedro Rodriguez.

   \par
   \vspace{1in}\raggedleft({\em Seraphina Goldfarb-Tarrant})
   \end{declaration}

% \dedication{stuff}

%% Finally, a dedication (this is optional -- uncomment the following line if
%% you want one).
% \dedication{To my mum
%% Create the table of contents
\tableofcontents

%% If you want a list of figures or tables, uncomment the appropriate line(s)
% \listoffigures
% \listoftables

\end{preliminary}
%\linenumbers

%!TEX root = ../thesis.tex
\chapter{Introduction} \label{chapter:introduction}
In the past decade, Natural Language Processing (NLP) systems have come to saturate everyday life. NLP has expanded from being used to translate webpages and recommend new videos to having inescapable reach; it is now also used to moderate social media content$^{\heartsuit}$ \citep{}, to generate answers to any user questions$^{\spadesuit}$ \citep{}, to track public opinion about products or politicians$^{\diamondsuit}$ \citep{}, to sort and filter resumes for potential new workers$^{\clubsuit}$ \citep{}, and myriad other applications.
%, and to track likelihood of committing a crime and recidivism rates$^{\star}$ \citep{}.  
This increase in scope and usage of NLP systems comes with many promises of efficiency, cost reduction, and even social good from scale--promises of (in order of the above examples) reducing hatespeech and aggression online$^{\heartsuit}$, of greater and easier access to information$^{\spadesuit}$, of direct and inexpensive democratic feedback for companies or policies$^{\diamondsuit}$, and of more efficient hiring practices$^{\clubsuit}$ that are less dependent on `who you know' and idiosyncratic instincts of a person.
%, and of better allocation of police resources or more objective, data-driven sentencing$^{\star}$. 

But this territorial expansion introduces many new harms that diminish these promises.  
The often referenced promise of mathematical objectivity---freeing us from human subjectivity, inconsistency, and biases---has proven to be mythical. 
At best, NLP systems learn and propagate these same biases, but with a veneer of objectivity that fosters over-reliance \citep{oneil2016weapons} and reduces accountability and recourse when data is incorrect and decisions go wrong. 
A large growing body of work analysing NLP systems has shown that they do not behave similarly and work equally well for different genders, races, nationalities, and other demographic groups. This is the standard definition of \textbf{fairness}, which we use throughout this thesis: these systems are not \textbf{fair}. 
So given that NLP systems, and the data, models, and optimisation and evaluation metrics they are composed of are \textit{not} inherently fair, we must analyse the ways they are not, so we know what to expect and can mitigate where possible. When NLP systems are not fair, companies and organisations using them (and people subjected to their outputs) are worse off than before automation systems, since this flawed system has now been scaled. An individual Human Resources manager may have flaws and biases, but they work for only one or a few companies and have time to read only so many resumes in a day. Some resumes will be sent to a different person, who may have different biases, preventing the inequities of the first person's views from being complete and consistent over a wide swath of potential jobs. When a flawed and biased NLP HR system is scaled, it does not sleep, get tired, or clock-off and can process as many thousands of resumes as time and compute allows. The same system is used by many companies. The very variability of human behaviour, and the inconsistencies in human decisionmaking that are often considered undesirable, limit the possible scope of each individual's (or even each company or organisation's) biases. This lack of scaling of humans is an accidental safeguard. An NLP system, in contrast, replicates the same biases to an unlimited extent, and whatever unfortunate minorities it is biased against will experience more widespread discrimination. This is the situation of the present day, and sets the scene for this research.

The research world noticed this, eventually. Recognition of fairness problems in NLP began in 2016 and has grown exponentially. By the time of writing, major conferences now have a dedicated track for fairness research\footnote{\url{https://aclrollingreview.org/cfp}}, encourage papers to self-declare potential hazards\footnote{Section A2 in \url{https://aclrollingreview.org/responsibleNLPresearch/}, and section 1c in \url{https://neurips.cc/Conferences/2021/PaperInformation/PaperChecklist}}, and have an ethics committee appointed to review potential fairness problems in any work\footnote{\url{https://www.aclweb.org/adminwiki/index.php?title=Formation_of_the_ACL_Ethics_Committee}}. Yet despite all the attention and effort expended on fairness issues in NLP, we as a community have made only such a small dent in known problems as to be aware of the magnitude of unaddressed fairness problems. Both discovering and addressing fairness problems in an NLP system remains extremely challenging. 

There are multiple ways that a system can be unfair. NLP systems are often not \textbf{allocationally fair}; they and have different accuracies and rates of false positives and false negatives for different demographics. A example such situation is when a toxicity detection system has much higher rates of false positives for text that is actually neutral or positive but contains terms about race, religion, or sexual orientation. In such as case, a sentence like \textit{ I am a gay man} can be flagged as toxic and censored, as was the case with Google's toxicity detection system in 2018 \citep{Dixon2018MeasuringAM}. NLP systems are also often not \textbf{representationally fair}; they reproduce and propagate negative stereotypes for minoritised demographics \citep{crawford_keynote}. For example, prominent generation systems will disproportionately describe women as taking carer roles, and portray racial minorities as criminals \citep{sheng-etal-2019-woman}. There is not even a consensus on how best to measure each type of unfairness. Most metrics used to measure fairness are ad-hoc and have not been standardised or analysed for \textbf{predictive validity}---their ability to predict actual fairness problems that will occur--or \textbf{concurrent validity}---their agreement with other metrics in use. We make some progress towards assessing predictive and concurrent validity of fairness metrics in Part~\ref{part:measurement}.  

Another challenge is that fairness issues can appear at almost any stage of building an NLP system \citep{suresh2021framework}. NLP papers commonly claim that `model biases reflect biases in data they were trained on' but this is both a such a gross oversimplification as to be unhelpful (how did the biases get into the data? Do imbalances in labels over different sensitive groups count, or do only stereotypes count?) as well as incorrect\footnote{I expect the prevalence of this statement is a way of shirking responsibility. It is the data's fault, and society's fault for creating the data, not the fault of the engineer or company}. All choices in the process of training an NLP model have been shown to affect the resulting bias. A resume filtering system can be trained on data in which humans made racist or sexist decisions (\textbf{historical} bias) and that bias will persist, be amplified, and scale, with the authority of an objective AI system behind it. This happened with Amazon's attempt at an AI for Human Resources.\footnote{ \url{https://www.reuters.com/article/us-amazon-com-jobs-automation-insight-idUSKCN1MK08G}.} A content moderation and toxicity detection system can be unfamiliar with non-prestige dialects and censor them incorrectly, as happened when tweets in African-American Vernacular English (AAVE) were incorrectly flagged as toxic speech \citep{sap-etal-2019-risk}. Even though AAVE is common in the world, it was not well-represented in data the model has seen, resulting in \textbf{sampling}, aka \textbf{representation}, bias). An NLP system often uses labelled data (for supervised learning) where the labels are a proxy for the task that is to be learnt -- e.g. in the resume filtering example \textit{was previously hired based on this resume} as a proxy for \textit{was suitable for the job}. That label can be a better or worse proxy for the desired task. This is \textbf{measurement} bias. These are only a few of the myriad ways that unfairness can enter a system, selected as examples as they are the types that I spend the most time examining below. There are more subtle ways that can make mitigation even more challenging, which we discuss in Part~\ref{chapter:background}.


% Remining things are aggregation (example from hardt) and evaluation (maybe example from my reality check paper) and also learning (the metric being optimised). 

%Fairness issues can arise also in fitting the same function to multiple groups that require different functions (aggregation bias) and in the metric used to evaluate (evaluation bias) \citep{hardt2016equality}.


%it has become a vital responsibilty for us as researchers to examine our models, algorithms, and data as to their behaviour for different demographic groups. Do they behave similarly and work equally well for different genders, different races, different nationalities? We need to know this information to ensure that the systems we build innovate, and improve society, rather than accelerating marginalisation and societal divisions and isolation. 

%These are the ways that unfairness can enter a system. 

%\sgtcomment{PICK UP HERE -- connect to the issues of *scaling* and of *not knowing how something gets into the system. Might need to connect transfer learning to it more smoothly -- since transfer learning is responsible for scale, and basically introduces a new source of all of these types of biases that is unexplored (a second set of data and of sampling and of measurement blah blah). It is also now the dominant paradigm (scale) that has been underexplored. And we want to know if these systems are better or worse than systems before using them, and we also want to pinpoint the source to enable us to mitigate it -- if we performed mitigation at the wrong stage it might not work}

An NLP system can contain one, many, or all of these sources of bias, and this bias can enter in via the data collection, dataset splits, learning objective, model architecture, model deployment choices (such as decoding hyperparameters or classifier thresholds). And worse still is that most of these choices are now made \textit{twice}.  Current scale in NLP is driven by \textbf{transfer learning}, where a model is trained on high resource task(s) or language(s) (e.g. unstructured web crawl text) and then ported to a lower resourced one (e.g. any supervised task requiring labels, like sentiment analysis) -- not necessarily objectively \textit{low resource}, but relatively lower resourced, i.e. with less data than the dominant task or language being used for transfer. 
%Within a language, this is usually done with a language model pre-trained on large scale web text and then applied to a supervised learning task requiring labelled data. Between languages, this is usually done between English and another language with less or no labelled data. 
It was already difficult to pinpoint where biases enter a system, and with transfer learning most systems are composed of multiple sets of training data, multiple objectives, multiple measurements. Transfer learning is now the dominant paradigm in NLP, but previous to the work in this thesis, fairness research considered only one of the two stages: the pre-training or the fine-tuning stage. If a language model that will later be used in our example resume filtering system (which we refer to as an \textbf{upstream model}) has been debiased with regard to gender, will the classifier on top of it (which we refer to as the \textbf{downstream model}) also be debiased, or not? If instead the classifier is debiased, is the language model also safe to use, or will bias then surface if the language model is used in another task, or directly without the classifier? We cannot answer these questions without studying the entire system and learning the relationship between upstream and downstream models. And without these answers,
%further complicating this already multiplex ecosystem of threats to fairness. It is now even harder to track and mitigate biases in NLP systems because they are multi-stage. 
bias mitigation methods or measurements are at best ineffective, and at worst misleading. With these answers we can apply effective bias mitigation strategies at the correct stage of the system, and we will understand the contribution of transfer learning to fairness in NLP systems and be informed as to whether systems are becoming better or worse as they scale. This understanding is a pre-requisite to effective work in NLP bias, and yet before the work in this thesis, the field had little knowledge of it.

So here, in the below, we explore a previously yet unstudied area of NLP fairness; how unfairness, having entered a system, persists and travels throughout it. 

We first focus, in Part~\ref{part:measurement}, on fairness measurement at different stages of transfer learning. No real research can be done without good measures, and we need an understanding of how measures of bias relate at different stages of transfer learning, since interventions are customarily applied at one stage. In Chapter~\ref{chapter:intrinsic_bias_metrics}, we study whether the most common \textbf{intrinsic} bias measurements--at the language model pre-training stage--are predictive of later downstream, or \textbf{extrinsic}, bias in two classification tasks in two languages. We find that they are not predictive, and that the widespread use of these measures has been leading to a false sense of progress in debiasing research. Most work was at the time done on only upstream models, and our work shows that we cannot tell whether debiasing efforts are propagating downstream. Our results show that more effort needs to be spent on measuring bias on the downstream task itself. Following this, in Chapter~\ref{chapter:gender_bias_probing}, we study the relationship of transfer learning measurements in the \textit{reverse} direction. Here we ask how a pre-trained upstream language model changes when different debiasing methods are applied downstream. We find that a new metric, based on information theoretic probing (also known as minimum description length (MDL) probing) \citep{} can, when applied to the pre-trained language model, differentiate between different downstream bias levels, and different downstream debiasing techniques, and show which are more effective. We find that this measure is predictive of how robust debiasing of the pre-trained language model is, and whether the debiasing will remain if that model is then used in another task. These two results together imply that the \textbf{geometry} (cosine or other distance measures, previously used as upstream metrics) of concepts in language model representation space does not reliably predict downstream bias, but the \textbf{extractability} of concepts (as measured by information theoretic codelengths) is better predictor. In that work, we also are the first to use a wide suite of ten downstream fairness metrics that refer to slightly different notions of fairness. We find that though they tend to track together, if we had naively used a subset of them, based on what was most popular for certain datasets, we might have come to a different conclusion. Different metrics are suitable for different applications and scenarios, and they do not always tell the same story. 

% Prior to the work in this dissertation, it has been unknown how fairness (and lack of) in an initial model relates to fairness in a system in which it is used. 

We then use our findings on measurement to conduct experiments addressing a broad question about how the use of transfer learning affects the fairness of a system. There is no previous work on this, but previous work on aspects of transfer learning leads to two competing possibilities of how transfer learning could impact fairness. Does transfer learning \textit{improve} fairness, because the additional data sources lead to overall better models that are better at modelling long tail phenomena \cite{} (and data on minorities is often long tail)? Or does the additional complexity bring in new or magnified undesirable biases, via one of the many mechanisms introduced above? 
%We study this for transfer learning between tasks within one language, and also for transfer learning between different languages. Within one language, will a language model that has been trained on raw text data that underrepresents women in prestige careers in STEM also fail to appropriately classify women's biographies into STEM roles \citep{biosbias} and incorrectly filter women's resumes for STEM positions? 

In Part~\ref{part:crosslingual} we pick a task---sentiment analysis, which we selected since this task enables us to test in a number of languages---and study this effect for transfer learning between \textit{tasks/objectives} (the current dominant NLP paradigm, which we will sometimes refer to as monolingual transfer learning to distinguish it) and transfer learning between \textit{languages}, called multilingual or crosslingual transfer learning (used interchangeably but the field and by us). Prior to our first investigation, previous work had shown that language models trained on unstructured text have gender and racial biases \citep{bolukbasi, Caliskan2017SemanticsDA, zhao-etal-2019-gender, zhao-etal-2020-gender, sheng-etal-2019-woman}. So we asked, will this carry through in monolingual transfer learning and cause gender and racial biases to appear or increase in a downstream sentiment model, beyond what can be attributed to the downstream training data? For instance, let's say that an upstream language model has learnt to associate conventionally negative attributes with certain minorities, such as to represent gay men as doing drugs, and black men as pimps (examples from \citet{sheng-etal-2019-woman}). Will a sentiment classifier built on this upstream language model also associate negative sentiment with gay and black men, \textit{even if} there is little or no data about gay and black men in the sentiment training data? 
Or will that bias be overridden or lost, either because the role of the classifier is strong enough to disregard that, or because the now larger and more expressive system can generalise better to other positive association involving black and gay men, such as stars in politics and arts, or affirmational personal stories, such as those in \cite{Dixon2018MeasuringAM}? We find that, overall, the additional stability from transfer learning is helpful in a resource constrained setting (i.e. one in which you cannot gather more annotated sentiment data), and this effect is enough to reduce overall gender and racial biases (despite new negative associations having been introduced).

We also study this effect for transfer learning between languages, or \textbf{cross-lingual transfer learning}. In this setting, not only can an upstream model learn biases from multiple data sources, but also from multiple languages. Exactly how much information cross-lingual transfer learning shares across languages is not well understood and there are some contradictory empirical studies \citep{}. So we ask, in cross-lingual transfer learning, if a language model has learnt harmful stereotypes in one language, can those negative associations carry across languages? In the above example where a model has learnt negative associations \textit{in English} about black and gay men, will a classifier in Japanese have these same associations, if they do not occur in Japanese? Can the collision between competing stereotypes in different languages weaken them, and in effect fight bias with bias? \citep{stanovsky-etal-2019-evaluating}. Can anything be done in the initial task before transfer, to ensure better outcomes in the second task? We find that, contrary to what we found in monolingual transfer learning, cross-lingual transfer learning tends to (with exceptions) exacerbate biases, though this effect can be mitigated with distilled/compressed models with little loss in performance. 

In Part \ref{part:generation}, we look at a third type of system: retrieval augmented generation, which presents an inversion of the standard transfer learning setup. In the standard setup,  a language model feeds into a classifier, and in retrieval augmented generation, the classifier selects source documents to answer a query, and this feeds into a language model, which conditions on those documents to generate an answer. This inverted system allows us to also ask the reverse question: if a language model has learnt problematic associations and stereotypes, can these be counteracted by conditioning on source documents? For instance, if a language model generates results about women predominantly in low-prestige roles, will it change this if it is conditioned on source documents about female CEOs and doctors? Or is it more likely to ignore the source information in this case then in the case of male CEOs and doctors? Or, as a third option, the retriever itself is biased, and doesn't select documents about female CEOs, so we never even get to that point?
However, prior to our work, not only was there no research examining how fairness flows between retrieval models and generative language models, there was little research analysing neural retrievers at all. So we began by asking the sub-question, inspired by all our work in Parts~\ref{part:measurement} and \ref{part:crosslingual}: a retriever representation is necessarily a compression of a document, so what information is actually in this representation, such that 
 a language model can condition on it? (Recall Chapter~\ref{chapter:gender_bias_probing} where information in a representation as measured by information theoretic probing is most predictive of bias). Is information about demographics--gender, race, etc--in a retriever representation predictive of allocational bias in retrieved results? That is, does a retriever with stronger information about gender pick documents about gender more unequally? We do a case study in allocational gender bias and find that, though retrievers quite strongly encode gender in their representations, allocational bias is not attributable to the representations themselves. This bias persists even when we remove gender from the representation, meaning that it comes from either the composition of the corpus or the queries themselves. We leave completing the high level question of studying what happens when a language model uses these representations to future work. 

 We conclude with a summary of our contributions, and with a set of recommendations in light of our findings.

\section{Contributions}
We make contributions to three broad categories: 
\begin{enumerate}
    \item More meaningful and reliable \textbf{measurement} of fairness in language models
    \item Analysis of how \textbf{transfer learning} affects fairness
    \item Analysis of fairness in \textbf{retrieval-augmented generation}    
\end{enumerate}

\subsection{Measurement} 
\textbf{Chapter~\ref{chapter:intrinsic_bias_metrics}}

We did the first study evaluating whether the most commonly used fairness metric for upstream language models correlated with downstream fairness. At the time, upstream only studies comprised one third of fairness research, making this a vital question. We examined the relationship between upstream and downstream metrics across a broad set of experimental conditions: two types of bias (gender, racial), two different tasks (coreference resolution and hatespeech detection), two different languages (English and Spanish), two common embedding algorithms (fastText and word2vec), two common methods of debiasing (preprocessing training data, and post-processing on representations), and two downstream fairness metrics (difference in precision and difference in recall). This is a much broader scope than most fairness research at the time this was done. We found that the common upstream metric, based on cosine similarity, was not predictive of downstream bias. This changed the focus of the fairness field toward evaluating bias downstream and finding upstream metrics that are more predictive. Our work has inspired follow up studies examining the predictive validity of fairness metrics \citep{cao-etal-2022-intrinsic, others?}, which further extend and corroborate our findings. 

\textbf{My contribution:} I designed the research agenda: I envisioned the research question and wrote a research plan document with methods, goals, metrics, and a literature review. I recruited and subsequently supervised three MSc students who implemented pipelines for three different systems and did initial investigations into the correlation between intrinsic and extrinsic metrics for their MSc theses. I gathered these pipelines together, modified and extended them, ran experiments, and wrote and presented the paper, with the help of Adam. 

\textbf{Chapter~\ref{chapter:gender_bias_probing}}

We also did the first study investigating how debiasing \textit{downstream} (rather than upstream) affects language model (upstream) representations. We focused on gender bias in English and considered two common transformer models, two tasks (coreference resolution, biography classification), three debiasing methods, two different intrinsic metrics (a contextual extension of the cosine similarity metric from the previous work and a new one, MDL compression, that we proposed adapted from \citet{voita-titov-2020-information}), and ten downstream fairness metrics. We found that our new metric was predictive of whether the upstream model had been successfully debiased, and correlated well with most downstream metrics. We also found that not all downstream fairness metrics correlated to each other, highlighting the importance of not relying overly much on one metric. 

\textbf{My contribution:} I was second author on this paper, assisting the first author Hadas Orgad who proposed this extension of the work in Chapter~\ref{chapter:intrinsic_bias_metrics}. I implemented some of the metrics on intrinsic analysis of language model representations, implemented the additional extrinsic fairness metrics (which are now open-sourced), and co-wrote the paper with Hadas Orgad and Yonatan Belinkov.

\subsection{Transfer Learning}
\textbf{Chapters~\ref{chapter:multilingual_sentiment_analysis} and \ref{chapter:multilingual_sentiment_analysis_pt2}}

We did the first research on the effect of both standard (monolingual) transfer learning and cross-lingual transfer learning on gender and racial biases in sentiment analysis. We first examined whether, for five languages (Japanese, Chinese, Spanish, German, English) monolingual transfer learning (via pre-trained models) changed the biases in sentiment analysis systems. We then ran similar experiments for the much more complex setup of multilingual transfer learning (via multilingual models and via cross-lingual labelled data). We found that, though the story is reasonably complex, cross-lingual transfer learning can increase bias even in unexpected cases such as culturally specific racial biases. Monolingual transfer learning usually reduces biases, even though the training data used for transfer contains new biases. It stabilises the model and that effect outweighs bad content learnt in pre-training. 

\textbf{My contribution:} I did this project almost entirely on my own save for the writing, which my supervisors Adam and Björn assisted with greatly. I designed the research question and outlined the project, developed and programmed the experiment framework, found training data, created evaluation data (with the help of native speakers of each language) and wrote up the results (with the help of my supervisors). I received regular weekly consultation from a team at Amazon Barcelona for the first of the two projects.

\subsection{Retrievers}
\textbf{Chapter~\ref{chapter:contrievers}}

We did the first analysis of the properties of \textbf{Dense Retrievers} (as contrasted with sparse TF-IDF based approaches), which are the basic component of retrieval-augmented generation systems. Knowing what information is in a retrieved representation is a pre-requisite to analysing how the retriever influences a downstream generative language model, but there was previously no work applying analysis or interpretability methods to retrievers. Dense retrievers are initialised from a pre-trained language model, and then further trained and adapter to excel at determining the relevance of a document representation to a given query, such as returning all the articles about Finnish prime ministers given the query, `Who is the prime minister of Finland?'. 
We analysed how the information captured in a representation differs for a retriever vs. the language model it was initialised from. We used information theoretic probing (based on the results in Chapter~\ref{chapter:gender_bias_probing} that is was predictive of bias) to analyse how extractable two features were from a representation: topic of a passage and gender of a subject. We analysed how these correlated to raw performance and to allocational gender bias. We found that gender extractability did correlate to performance on gender related questions and allocational gender bias, but that allocational gender bias persisted even when gender information was erased, meaning it was not attributable to the representation itself. We thus show another case when an entire system has to be considered in debiasing an NLP system.

\textbf{My contribution:} I designed the research question, in collaboration with Patrick Lewis. I made the research project plan, chose datasets and interpretability methods, and wrote all pipeline code and ran all experiments, and finally wrote up the findings into a paper, with very minor proofreading from three colleagues.

\section{Recommendations}
In light of this body of research, we make the following recommendations. 

On \textbf{Measurement}, we recommend not to use geometric intrinsic measurements of bias (based on cosine-similarity like WEAT \citep{} and CEAT \citep{}), as they are not predictive of downstream behaviour. This is true regardless of whether they are applied to a non-contextual embedding like word2vec \citep{}, or to a language model like BERT \citep{} or RoBERTa \citep{} and company. These metrics \textit{are} good for studying human social biases via what is reflected in the data that trained the model, as was done in the original work of \citet{Caliskan2017SemanticsDA} that inspired the usage of this type of metric.\footnote{Though for this type of use case we note that RIPA \citep{} is likely better, or at the very least word frequencies need to be normalised for results to be valid.} But they are not good for predicting \textit{model} behaviour. We can tentatively recommend instead using information theoretic probing as an alternative and reliably predictive intrinsic metric. However, this recommendation comes with two limitations: we studied information theoretic probing only for \textit{allocational gender bias in English}. First, gender encoding differs greatly in different languages (more than other demographics) due to gender agreement systems, so these findings should be validated in more languages before being trusted beyond English. Second, even including English, other biases may not be stored the same way (for the same reason of the grammaticality of gender). So for other types of bias, no intrinsic metric has yet been validated and downstream metrics should still be used until more research has been done. Research on other options for intrinsic measurements is nascent, and we recommend always measuring fairness on a downstream task rather than in a language model when possible.
%\sgtcomment{should I say why I might expect gender bias to be an exception? A: it is much more strongly encoded in langauge than other demographic signals, save that given by dialect, and so this might affect the extractability to bias relationship}
We also recommend that downstream metrics be selected with reference to the desired system behaviour. This may seem simple, but few works in the NLP literature acknowledge this, despite that the suite of all downstream fairness metrics is provably not mutually satisfiable, so you do actually have to pick one as a constraint. Different downstream metrics mean different things, and debiasing efforts often will only make sense for some metrics. Equalised false positive rates make more sense in the context of content moderation or toxicity, where the risk is censorship, equalised false negative rates make more sense for resume screening where the risk is excluding people from the potential to interview. In NLP, we often try to avoid making normative decisions about the world that our models will be embedded in; it is a messy and complex world, even more so than our data. Part of the brittleness and unreliability of bias evaluations and bias metrics--poor predictive and concurrent validity--is that researchers don't think these through and make them explicit. Each debiasing method only make sense for some type of bias, and our better intrinsic metric from Chapter~\ref{chapter:gender_bias_probing} still only correlates with most extrinsic measures, there is a family of measures that it does not work for. Fairness researcher do need to engage with the world they are imagining and how it should work. All fairness work contains an assertion like this, and if left implicit, it can be scientifically messy. So we recommend that researchers make explicit, reasoned choices about why they use the metrics that they do. 

On \textbf{Transfer learning}, we recommend to use monolingual transfer learning (also called pre-training) when working with less data, at least for classification. We tested sentiment classification in three language families, so we expect our findings to hold for all similar tasks, but cannot claim to generalise to generative tasks.
However, we recommend to take more care when using cross-lingual transfer learning, as it risks introducing new biases into the target language from other language data. When cross-lingual transfer learning is used, we recommend using distilled cross-lingual models, as we found distilled models to have nearly equivalent performance and much lower bias overall than their full-size counterparts. We recommend also the use of two of our analytical methods: causal or counterfactual evaluations, combined with a granular heatmap based analysis of the results.

On \textbf{Retrievers}, we recommend to analyse the entire system: corpus, queries, and model representations, as our work shows that a model constrained to have perfectly fair representation may still create an unfair system because of the other components. From the extensive experiments on random seed initialisations in this section, and the smaller scale experiments in the previous, we also recommend to test models based on a large number of random initialisations. We found this to have a disproportionate effect on model fairness and model performance both. In cases where trustworthy evaluations are available, ones which are faithful to a use case and which generalise, they can be used to select a seed with better generalisation properties for fairness, and this difference can exceed the difference from any common debiasing approaches or interventions. In cases where this is not possible, we recommend using majority voting across three to five random seeds, to minimise by seed variance. 

%Adam reference: https://matt.might.net/articles/advice-for-phd-thesis-proposals/

% NOTES TO SELF:

% First part is methodological -- this is how we measure the process, and then we apply this to other things.


\chapter{Background}\label{chapter:background}

The following sections give requisite background information that common across all research in this thesis. Background that is relevant to only an individual work will appear before that work.

\section{Defining Fairness}
Fairness is a relatively recent interest in the field of Machine Learning/AI research, and as such it suffers from lack of standardisation in both definitions and methods of measurement. Much to its detriment, as any work must first define what it means by the term, and as idiomatic definitions and measurements across different works make meta-analysis difficult, and can hide contradictory results across different experiments. Fairness began to gain attention in AI in 2016, following the publication of the popular book \textit{Weapons of Math Destruction} \citep{oneil2016weapons}, which detailed all the ways that Machine Learning systems propagate injustice and create inequity in society, and the NeurIPS research paper \textit{Man is to Computer Programmer as Woman is to Homemaker? Debiasing Word Embeddings} \citep{bolukbasi}, which showed, via evocative word analogies from neural word embeddings, how career based gender bias was learnt by these systems, even when trained on relatively innocuous (for the internet) content like Google News. Machine Learning was shocked and galvanised by these works, though it was, as usual with surprising new discoveries, late to a party that other fields had been aware of for some time, as education and hiring had been looking at statistical fairness for the previous half a century \citep{hutchinson_mitchell_2019}. Since then, major NLP and AI conferences have added entire Ethics tracks, formed Ethics commmittees to review papers flagged for potential ethical concerns, encouraged ethics statements to be included in each work, and multiple fairness focused workshops have sprung up at each conference. But definitions and methods are still being solidified.

Broadly, fairness can be categorised as one of two types, as mentioned in the Introduction. \textbf{Allocational fairness}, which requires that systems perform equivalently for different individuals, regardless of demographics, and \textbf{representational fairness}, which requires that the systems represents different demographics with equal dignity. How exactly "equivalently" is measured is what determines and differentiates the available fairness metrics. For intuition, an example of allocational unfairness is how Automated Speech Recognition (ASR), which is now absolutely everywhere, has different error rates for different dialects of English, with increasing error the farther that speaker is from a white male in his 20s from California \citep{tatman17_interspeech}. An example of representational unfairness is how generative language models disproportionately generate text about black men as criminals and gay men as druggies, as compared to white men \citep{sheng-etal-2019-woman}. It is possible to have some blurry boundaries between these two, as in \citet{zhao-etal-2017-men}, which found that image captioning models were inaccurate for counter-stereotypical gender activites, like men shopping or cleaning, and women riding motorcycles or programming/gaming. This work frames the fairness issues as allocative (research tends to pick one of the two areas), but it could also be considered representational, as any application, e.g. an image search, that relies on a model with these errors will produce only stereotypical images.

\section{Measuring Fairness}
\label{sec:measuring_fairness}
\sgtcomment{I should probably introduce invariance here rather than just performance gap, since I use invariance as as a measurement in Part II. the difference between this is  causality (interventional vs. observational) which I should establish and explain the pros and cons}

\textbf{Notation:} In all fairness metric definitions contained in this work, let $a \in A$ be the demographic variable in question, where $A = \{privileged, minoritised\}$ group, such as $\{male, female\}$ or $\{native, immigrant\}$\footnote{An obvious limitation of this is that privileged and minoritised is binary, and tends to be true of fairness work, including work in this thesis. There is insufficient work on extending fairness metrics and constraints to multiclass, either theoretically or empirically.} In classification tasks (all tasks until Part~\ref{part:generation}) let $Y$ be the true label, $\hat{Y}$ be the predicted label, and $R$ be the classifier score (which enables analysis independent of classifier threshold). 

\textbf{Representational fairness} has no codified metrics of measurement in NLP, which is perhaps the most obvious of the many areas where NLP could learn from sociology and psychology and has failed to, for they have been measuring representational fairness in media for quite some time \citep{}. The field largely neglected to measure this until \citet{sheng-etal-2019-woman}, which proposed using a classifier to detect \textit{regard} for the subject of a passage in open domain generation, and found GPT-2 to systematically generate content causing lower regard when generating about women, African-Americans, and gays. This work is conceptually satisfying, and important, but difficult to expand due to the reliance on the classifier, which is limited to English and can become out of date over time as language drifts, so there have been a few follow up replications of this work but not many \citep{goldfarb-tarrant-etal-2023-prompt}. Other work on representational fairness that exists focuses on discovery of language model stereotypes, which makes up the majority of bias work done on generative models today \citep{goldfarb-tarrant-etal-2023-prompt} as measuring allocational fairness is not straightforward in generation. However, the most prevalent approaches to stereotype measurement, from two benchmark datasets, have been shown to be so flawed in construction as to be essentially meaningless \citep{blodgett-etal-2021-stereotyping}. As a result, this work focuses on only \textbf{allocational fairness}.

Measurement of allocational fairness is best overviewed by \citet{hutchinson_mitchell_2019} and \citet{barocas-hardt-narayanan}. At a high level, allocational fairness can be measured as \textbf{individual fairness}, which answers the question "are the results for similar individuals equivalent" and \textbf{group fairness}, or "is the performance for demographic subgroups equivalent". In the former, the work lies in defining the similarity function, in the latter, the work lies in selecting the demographic slices, and in choosing the performance measure. Choosing the demographic slices does not get much attention in NLP literature; there are a few nods to intersectionality \citep{} and to unsupervised demographic group discovery \citep{zhao-chang-2020-logan} and otherwise works assume that demographic groups are given, gold, and that discrimination against different demographic axes is independent-- ie discrimination against women can be treated entirely separately from against African Americans. In real life, this is patently false \citep{some intersectionality thing}. It remains unexamined how well this assumption works for Machine Learning. Most ML and NLP work uses group fairness, and compares the difference in performance across subgroups.  In classification, performance used is sometimes difference in F1 \citep{zhao-etal-2018-gender} but is often more granular, such as \textbf{equalised odds} \citep{hardt2016equality} which enforces equal false-positive rates (FPR) or true-positive rates (TPR) across groups.
\begin{align}\label{eq:fpr}
    P(\hat{Y} = 1, A=a, Y=0) (FPR)
\end{align}
\begin{align}\label{eq:tpr}
    P(\hat{Y} = 1, A=a, Y=1) (TPR)
\end{align}
where \ref{eq:fpr} and \ref{eq:tpr} are equal $\forall a \in A$.

Note that the second constraint \ref{eq:tpr} is equivalent to recall, as recall can be expressed the same way:
\begin{align}\label{eq:recall}
    \frac{\hat{Y} = 1 | Y=1}{(\hat{Y} = 1 | Y=1) + \hat{Y} = 0 | Y=1}
\end{align}.
The second constraint is often used in isolation and is terms \textbf{equality of opportunity} \citep{hardt2016equality} as a relaxation of \textbf{equalised odds}. 

Occasionally some works include related but different group fairness metrics, discussed in \citet{barocas-hardt-narayanan}, such as \textbf{independence}, \textbf{separation}, and (rarely) \textbf{sufficiency}. 

We use independence and separation, as well as the others, in \ref{chapter:gender_bias_probing}, to use the broadest number of metrics possible in order to establish the relationship between language model representations and application fairness metrics (also called \textbf{downstream} or \textbf{extrinsic} metrics). 

We consider a broad set of metrics because fairness metrics should be chosen based on the tasks in question. Different notions of fairness are in tension with each other, and are provably mutually unsatisfiable \citet{barocas-hardt-narayanan}. So fairness metrics need to be chosen based on the application and what makes most sense. Choices of fairness metrics involve a normative judgment, whether implicit or explicit. This is too often left implicit, and is often made based on what some prior similar work has used, for comparison, even if a different metric is both able to be used and would be more suited \citep{orgad-belinkov-2022-choose}.

These metrics are all ones that are applied on some downstream task. In Part~\ref{part:measurement} we analyse the relationship of these allocational fairness metrics to a number of novel measurements proposed in the NLP literature that can be applied to just representations. We will not survey them here, as what we use is specific to each work.

It seems potentially obvious to state, but a desirable characteristic of a measurement of fairness is that it a) accurately measures the concept that it purports to measure and b) has a reliable relationship to real world fairness. When a and b are true, the measurement has \textit{construct validity} -- a multi-faceted concept in the field of measurement modelling from the social sciences \citep{jacobsandwallach}, that attempts to define and make explicit the gaps between conceptualisation (e.g. my model should not discriminate based on race) and operationalisation (e.g. the performance gap between different racial groups, as identified by dialect identification). a) corresponds to \textit{content validity} and b to \textit{predictive validity}. Much of the work in Part~\ref{part:measurement} is motivated by our observation that these types of validity had not been examined and were assumed to be true. We thus set out to test and improve upon these types of validity.

In the first work on measurement (\ref{chapter:intrinsic_bias_metrics}), we focus on gaps in precision and recall, as previous work upon which we built our analysis used F1 \citep{zhao-etal-2018-gender}, and factoring them out gives both more granular analysis and also comparability to the equality of opportunity measure \citep{hardt2016equality}. In the second, we use the full suite of possible metrics. In Part~\ref{part:crosslingual} we use don't use a subgroup metric, but instead use counterfactual examples that perturb one demographic variable, where we make an invariance assumption that values should not change under this perturbation, and the magnitude of the change is our metric. This method does not fit cleanly into individual or subgroup fairness, as it can be analysed on an individual example (which we do) but those examples have also been constructed to stand in for a demographic. E.g. in the counterfactual example: \textit{I made her feel relieved} vs. \textit{I made him feel relieved}, \textit{her} and \textit{him} are individual instances of bias, but also are stand-ins for the concept of gender. In Part~\ref{part:generation} we measure retrieval rather than classification, so we use performance gap in the most common retrieval metric.  

\section{Common Approaches to Debiasing}
Fairness literature, as well as \textit{measuring} bias, will often propose methods of \textbf{debiasing}. Different debiasing methods proliferate, but most new methods do not get widespread adoption, since they fail to build trust. Debiasing methods tend to be proven in only quite constrained settings, on only one or two models, only in English, and on a limited number of tasks. \footnote{There is probably also some effect on the bias in publishing on encouraging novelty that new works tend to propose new methods rather than verifying existing methods, such that we have zillions of methods that no one wants to use}. This thesis therefore focuses on analysis, and does not propose any new methods. However, we will briefly survey existing methods that are used in Chapters~\ref{chapter:intrinsic_bias_metrics}, \ref{chapter:gender_bias_probing} and \ref{part:generation}.

Debiasing approaches fall into high level categories of where they occur in the lifecycle of training an NLP model: pre-, mid (during), and post. \textbf{Preprocessing}\footnote{I use the term preprocessing rather than pretraining to distinguish from the now common terminology of pretraining/finetuning} approaches involve a processing step that modifies data before training the model to reduce signal that can cause bias. For example, if a system used for resume filtering should be debiased with regard to binary gender, the data can be processed such that there is an equal occurrence of gender signifiers (pronouns, names, other words that encode gender information)\footnote{This is never easy to do fully, but can be quite successful in English with relatively coarse processing. It is not so easy in languages with much more gender marking, and this area is heavily underresearched. \citet{} looks into using morphological analysers for this.} with words that indicate profession or career information. This can be done on unstructured webtext that is used to learn embeddings or train language models \citep{} or on labelled data that is used for supervised finetuning \citep{}. These are usually known as \textbf{dataset balancing}, and differences in dataset balancing approaches stem from the method of changing initial data and the axes along which the data is balanced (gender/profession, race/toxicity, religion/sentiment, etc). If there is excess data (more commonly true of unstructured text), it can be subsampled such that there is less data and it is more balanced \citep{}. Other approaches are to oversample data such that some data is repeated \citep{} in order to essentially overweight those examples because that they occur more frequently in training data. This suffers from lack of diversity in the minority class, so some works opt to create synthetic data \citep{}. While this approach is most reliable, it is only available to practitioners who actually train models, which was always a small class and increasingly smaller as models continue to scale. There is also very little work that tries to balance multiple axes at once [Is there any?], there are a few works on intersectionality (which is not the same thing but related) \citep{} but even those are rare. And the difficulty of subsampling for instance would likely become infeasible when balancing multiple biases, which is essentially never done in research, but is a perfectly reasonable desire in real life applications, and for many cases is a required feature, as very few regulations and systems of integrity specify just \textit{one} minoritised group. This is clearly important future work.

Debiasing can be done in \textit{postprocessing} as well, generally on representations, though there is some preliminary work investigating decoding \citep{sheng-etal-2021-societal}. Both approaches are conceptually more complex than preprocessing, but do not require retraining a potentially very large and expensive language model. This enables postprocessing approaches to be done by parties further downstream than those that pretrain language models. This is crucial for appropriate debiasing, as it is then most connected to a downstream application, which we show is necessary in Chapter~\ref{chapter:intrinsic_bias_metrics}. It also allows more iteration and experimentation without extensive compute. \citet{ravfogel-etal-2020-null} operate on representations via nullspace projection -- they learn a linear classifier for binary gender, and then project language model representations onto the nullspace of that classifier. \citet{iskander-etal-2023-shielded} extend this method to remove non-linearly encoded information. We use these methods for analysis of the impact of demographics (rather than debiasing) in Chapter~\ref{part:generation}. The other methods of debiasing representations operate on individual words and groups of words that stand in for concepts: \citet{mrksic-etal-2017-semantic} pushes words together or away from each other in representation space; we use this method in Chapter~\ref{chapter:intrinsic_bias_metrics}. More recently a number of works use  `model-editing'  \citep{meng2022locating}, where individual neuron values can be changes in order to change one specific string. This has some issues with scaling to a full demographic (edits are granular) but would be a promising new direction for very targeted interventions.

%\citet{huang-etal-2020-reducing} takes this a step farther and regularises output to have similar sentiment under a counterfactual

Debiasing during training, via constraints or costs to the learning method \citep{zhao-etal-2017-men}, is less commonly done, perhaps partly perhaps because it contains the disadvantages of both preprocessing and postprocessing -- it is conceptually more complex and requires tuning hyperparameters (as does postprocessing) but it also requires retraining a model. It also is made more complex by transfer learning, as it is unclear whether to do it at one or both stages. We leave out this kind of debiasing from all of our analysis.

Recent hype around generative Language Model fairness focuses on a second stage training process, often called `alignment', which refers to the idea that human morality (it is never specified which human or which morality) can be instilled in a model by essentially incorporating a fine-tuning with ranking loss stage where the objective is for the model to be able to predict the correct ranked order of generations that differ with respect to desirable and undesirable properties, from grammaticality to length to factuality to fairness. The majority of this thesis predates the alignment trend, and very little deals explicitly with generation, so we do not use any of these techniques. Much of the alignment work has origins more in robotics than in fairness (as it is usually (though not always) applied via reinforcement learning). However, our work does have interesting similarities. In Chapter~\ref{chapter:gender_bias_probing} we measure fairness via differences in distributions for different demographics, via KL or Wasserstein distance, one of the standard ways to measure it (\S \ref{sec:measuring_fairness}). \citet{korbak2022on} show that Reinforcement Learning from Human Feedback (RLHF), the most common method of alignment, can be equivalent to distribution matching. So when given an appropriate setup, RLHF could (theoretically) be directly optimising for a fairness constraint.


%\subsection{Causal interventions for analysis}
%This is not bias specific, but is relevant for both parts 2 and 3 so I think it belongs up here


\section{Transfer Learning}
Transfer learning is, at the point of this thesis being written, so common that it is not generally specified anymore, but is the unstated default. Back when the research in this thesis was in its infancy, both the fields of fairness and of transfer learning were very small but growing exponentially. It seems that transfer learning has won, as almost everything is transfer learning (though fairness is no small field anymore either, and it is now common for work that is about any given topic to include a section on fairness evaluation). 
The central premise of transfer learning is that it doesn't make sense to start from a tabula rasa randomly initialised weight matrix every time you want to learn a new task, but that many of the concepts necessary to learn for one NLP task may be in common with another. For instance, toxicity detection and sentiment analysis both require knowledge of basic sentence structure, nouns, verbs, and negative connotations of different words, and so knowledge from one should be able to augment knowledge from the other. Even more dissimilar tasks like toxicity detection and coreference resolution still require a similar underlying knowledge of sentence structure. Early work in transfer learning often sought to transfer from task to task like this; this approach is called \textit{domain adaptation}, or when done simultaneously rather than in sequence \textit{multi-task learning} \citep{ruder2019transfer}. This is now less common, and the field of transfer learning has coalesced into sequential pretraining of a language model on unlabelled text, followed by task specific fine-tuning, which is the approach taken in this work in Part~\ref{part:measurement}. This leverages the extremely large amount of unstructured text available to build high quality representations of language that can then serve as initialisations for any downstream language task. 

We also use a variant of transfer learning in our analysis that combines aspects of both the pretrain-finetune paradigm and the domain adaptation paradigm, when we do cross-lingual transfer in Part~\ref{part:crosslingual}. In cross-lingual transfer learning, the language model pretraining stage is multilingual (contains text in a variety of languages) and the fine-tuning stage is in a high-resource language (usually English) that doesn't match the target language at inference time. This is strictly called \textit{zero-shot cross-lingual transfer} (ZS-XLT)since the model has never seen labelled data in the target language. There also exists \textit{few-shot cross-lingual transfer} (FS-XLT), where the high-resource fine-tuning is continued with a few examples in the target language (often only hundreds). FS-XLT has been shown to generally perform better than ZS-XLT for relatively low additional annotation cost \citep{}, but we use exclusively ZS-XLT in this work, since FS-XLT adds many additional layers of tuning and variability to the already complex landscape of cross-lingual transfer and the small additional bump in performance was not necessary for our analysis.

Early transfer learning varied between whether to `freeze' the language model after the first stage, and just learn whatever new parameters need to be added for the desired final task output space (e.g. the classifier or coreference model or etc that takes in the representations) or to continue to train the language model along with the second task, to further refine the representations to best fit the task specific needs. \sgtcomment{I think the idea for this was whether it would generalise better if you froze? But I don't remember actually.} In this work we use both methods, depending on what is most standard for the task or what enabled ease of analysis. We always specify in each work's respective methodology.  


%Note: Can do a subsec "fairness in pretrained models in transfer", "fairness in downstream models in transfer" and then the lack of anything doing both will be helpful motivation

\section{Fairness as Dataset Artifacts or Failure to Generalise}
\label{sec:fairness_as_other_fields}
The allocational fairness measurements we use (\S \ref{sec:measuring_fairness}) can have a number of different causes. For instance, racial bias in toxicity detection can come from labelled training data correlating African American dialect (AAVE) features with toxic content, as a result of an error or a bias in annotation \citep{sap-etal-2019-risk}. But allocational racial bias could also come from insufficient training data in African American dialect , resulting in higher error rates from that group, as \citet{tatman17_interspeech} measure for automatic captioning. Most work does not address this difference or disentangle these two causes, and lump both under "bias". It is unsurprising then that so much work fails to have predictive validity: correcting an anti-AAVE stereotype may not help allocational bias in a model if the root cause was simply that it modelled AAVE poorly. I will refer to these as dataset bias and generalisation failures. Even dataset bias can actually be split into two types of causes conceptually: dataset biases that replicate historical biases (most previous engineers hired were men, and so the dataset of successful resumes is mostly male resumes) and dataset biases from 'dataset artifacts' such as a correlation between length of resumes in lines and 'hire' label, combined with a notable difference in resume length between different genders. These two are detectable via similar methods and in some sense they are the same -- they are a shortcut to a task supported by a dataset -- but they may be differently anticipated by humans, as one is predictable given knowledge of historic dataset biases, and the other is difficult to anticipate, and often downright hilarious, as when NLI contradiction could be largely predicted by the presence of words about cats \citep{gururangan-etal-2018-annotation}. They do belong in the same category as far as causal effects, but the conceptual difference can have an impact on discoverability.

The collapse of these two causes into one measure is not necessarily bad, since the fairness effect in an application is equivalent, though as above it can cause some confusion. It would be beneficial for the field to regularly develop ways to split out these two causes (though causal analysis is of course challenging as it requires interventional studies which often take much more time and effort than observational). Splitting them out also shows the overlap between fairness work and other areas of NLP, and is thus valuable in its own right. Dataset bias work has significant overlap with work on dataset artifacts and on `shortcutting' \citep{geirhos2020shortcut}, generalisation failures have overlap with research on robustness and generalisation. This is rarely recognised or leveraged. For instance, AFLite is an algorithm developed to search a dataset for artifacts (such as `cat' and `contradiction') \citep{LeBras2020AdversarialFO} and then filter them, and comes from the dataset artifacts literature. LOGAN \citep{zhao-chang-2020-logan} is an algorithm for unsupervised discovery of social biases, from the fairness literature. They are implemented differently, AFLite is conceptually similar to k-fold validation with targeted sampling for artifacts, and LOGAN is a modification of k-means. But they can both be used to solve the same goal of finding slices of a dataset that exhibit strong imbalances based on a feature that should not have an imbalance.  Similarly, work on fairness showing that automatic captioning doesn't generalise well to accents beyond white Californian male accents \citep{tatman17_interspeech} and that facial recognition doesn't generalise to non-white skin \citep{buolamwini18a} has much conceptually in common with generalisation work showing that natural language inference doesn't generalise to new syntactic structures \citep{mccoy-etal-2020-berts}. The areas do not acknowledge each other currently, nor share mitigation or analysis techniques, but there is much room to do so.

Given this observation, in my work I attempt to take inspiration and techniques from these related fields and incorporate them into fairness research. In Chapter~\ref{chapter:gender_bias_probing} and Parts~\ref{part:crosslingual} and \ref{part:generation} we run all experiments on multiple random seed initialisations, and analyse the models separately by seed which is rarely done in fairness work, but which generalisation work has shown to be vital \citep{mccoy-etal-2020-berts, multiberts}. Our results show this is crucial, and different seeds do show drastically different fairness properties despite equivalent development set performance, just as was found in \citep{mccoy-etal-2020-berts}.

We hope that in future these fields have more dialogue and joint work. 

\part{Measuring the Relationship between Fairness in Pretraining and Fairness Downstream Applications}\label{part:measurement}

% Background for this probably talk about predictive vs. descriptive measures
\chapter{Intrinsic Bias Metrics Do Not Correlate with Application Bias}\label{chapter:intrinsic_bias_metrics}

\includepdf[pages=-]{pdfs/2021.acl-long.150.pdf}

\chapter{How Gender Debiasing Affects Internal Model Representations,
and Why It Matters}\label{chapter:gender_bias_probing}

Notes:

Talk about how it extends to contextual (and concurrent work that finds the same thing)

Talk about the importance of multiple different metrics and maybe why (though I might not have the experiments to back this up).

Then also talk about what it means about transfer -- we can see what a language model will carry with it to something else. 


Can extend some sections and remove others if not doing by publication. Think about which ones to include.

Say which parts of the work I did since second author. 
\includepdf[pages=-]{pdfs/2022.naacl-main.188.pdf}


\part{Fairness in Transfer across Languages}
\label{part:crosslingual}


Maybe consider adding label balance addendum
\chapter{A comparison of sentiment analysis with no transfer, monolingual transfer, and multilingual transfer}\label{chapter:x}

\includepdf[pages=-]{pdfs/bias_beyond_english_final.pdf}
\chapter{Cross-lingual Transfer in Sentiment Analysis}
\label{chapter:multilingual_sentiment_analysis_pt2}

The next work directly follows on from the previous results: the experiments were planned together and directly inform each other, despite being published separately. 

In the previous work, we created the resources needed to do these experiments, as it was the first work on fairness in language models across multiple language families. We found that there is an effect on fairness from transfer learning within one language. We can't disentangle the exact \textit{causes} of this effect from those experiments: whether it is information contained in the data, or the additional stability of the model from the addition of more data, though we hypothesise the latter (stability from more data) is the cause. Regardless of the causes, the findings are useful in practice under resource constraints, if less scientifically satisfying than if we had controlled all variables. 

In the following work, we examine the more complex setting of cross-lingual transfer in all the same languages, and again ask how this setting changes fairness outcomes. However, we set up our experiments to control as many variables as possible and establish causes, without pre-training all new models from scratch (that is, we limit ourselves to fine-tuning only, as the multilingual setup has already extremely many variables and requirements on compute resources). The full set of experimental variables we consider is:
\begin{itemize}
    \item Type of bias. We look at gender bias and racial/country of origin bias. We might expect these to have different patterns of cross-lingual transfer as gender is encoded in some languages in a way that race is not (via gender agreement) and as gender biases tend to be global and common across languages in a way that racial biases are not (the minoritised racial groups differ culture to culture).
    \item Mono vs. multilingual pre-training. We examine what happens when changing from monolingual to multilingual pretraining \textit{without} changing the fine-tuning data. This would not be none in practice in a production system, but enables us to isolate the two types of data (pretraining and fine-tuning) that usually change when going from a monolingual transfer to a cross-lingual transfer model. In our first experiments, we hold fine-tuning data constant for each language and change only pretraining data.
    \item Target language fine-tuning vs. transfer language fine-tuning. In a monolingual transfer setup, a model applied to Spanish as the target language will be fine-tuned on Spanish. In cross-lingual transfer, it will be fine-tuned in another language (in this case English) and then applied to Spanish. In these next experiments, we hold the pretrained multilingual model constant and change the fine-tuning data.
    \item Random seed. All experiments report the majority vote over five random seeds for the weight initialisation of the classifier and the data shuffle for fine-tuning. We initially did an analysis by individual random seed, and found them to differ so strongly that sometimes even polarity of the bias flipped: that is, for random seed A there would be anti-female bias and for random seed B there would be anti-male bias. We take majority vote to indicate what would be the most likely thing to happen for a random seed picked out of a hat.
    \item Distillation. We do the same set of experiments for full-size (100-150 million parameters) and for distilled models, which are approcimately half the number of parameters. These experiments make our results more applicable in practice, as distilled models are commonly used in combination with cross-lingual transfer as both are methods to deal with insufficient data or resources.
    %\item Label balance. 
\end{itemize}

There are nonetheless two experimental variables that we consider important but were unable to include. We do not look at the effect of modifying pretraining data: it is an important variable in the manifestation of social bias, but it is the most difficult to experiment on because training a model from scratch is so challenging. It is also the one least likely for developers of NLP systems to modify in practice, for that very reason. We also do not look at the effect of domain match/mismatch. In practice many sentiment systems have a domain mismatch, since sentiment training data tends to be from domains where sentiment can be determined from freely available metadata without an annotation effort: movie, restaurant, and product reviews. Our experiments reflect this domain mismatch by training on product reviews and using standard text at inference. However, results may differ for in-domain data, and it would be possible to also create a bias evaluation dataset from in-domain data and observe the differences.

As a result of these ablations, this work focuses on two types of causal tests. The evaluation dataset is based on counterfactual pairs, which are causal tests that answer the question not just \textit{what changed} (as an observational study does) but also \textit{why did it change} (a property of an interventional or causal study). This is a different \textit{why} than is answered when we do all of our experiments, which are themselves a different kind of counterfactual. The evaluation data holds everything fixed save the demographic variable, such that any change is attributable to the perturbation of the demographic variable. The many experimental scenarios hold everything fixed save a specific difference in model training, so that difference becomes the variable that can establish causality.
We also leverage the analytical method from the previous work on using a counterfactual confusion matrix to visually inspect patterns of bias. 

The many experiments thus answer whether the observed behaviour came from pretraining or fine tuning as best as possible. There is a limitation, which is that this setup is unable to isolate any interaction effects (which there almost certainly is because pretraining sets inductive biases). It also doesn't answer what about each step caused the change (what segment of data, what hyperparameter). We are unaware of any work that can manage these questions, but we do want to call out that though this work is very rigorous on causal attribution, it is still able to establish causality to only a limited extent and far more research into this area is needed for it to be understood. We released all the models in the hopes that other researchers will do some of this work. 


\includepdf[pages=-]{pdfs/cross_lingual_transfer_final.pdf}

%\chapter{How does grammatical gender affect representations when it has no relationship to natural gender}\label{chapter:x}


\chapter{A chaper}\label{chapter:x}

\chapter{MultiContrievers: Analysis of Dense Retriever Representations}
\label{chapter:contrievers}
\includepdf[pages=-]{pdfs/probing_multi_contrievers_ARR_deanon.pdf}

\part{Conclusion}
\label{part:conclusion}
\chapter{Conclusions, Questions, and the Present Day}\label{chapter:conclusion}

\say{\textit{For every subtle and complicated question, there is a perfectly simple and straightforward answer, which is wrong.}}  —H.L. Mencken

In this section I will condense the takeaways from my work, then expand them into the new questions they pose to the field. I'll discuss implications for future work, do a bit more reflection than in each individual piece of research, made possible by the aggregation of all the my work. I will also include a less traditionally academic section, and relate my work to the present day mania for Large Language models (LLMs), generative AI, and scaling.

\section{Takeaways}
This body of work has focused on \textbf{measurement of fairness}, and then, with respect to those measurements, on \textbf{analysis of monolingual and cross-lingual transfer}, and finally analysis of \textbf{dense retrievers}. It contributes to enriching our understanding of fairness within a multi-part system, which was previously poorly understood. This poverty was much more marked at the commencement of this work than today, but does still persist. Fairness and interpretability research have grown exponentially, but so has the field as a whole, at an equal pace.\footnote{Fairness work has grown exponentially, but so have ACL and NeurIPS, both with about 40-50\% growth in submissions year over year, \url{https://aclweb.org/aclwiki/Conference_acceptance_rates} and \url{https://medium.com/criteo-engineering/neurips-2020-comprehensive-analysis-of-authors-organizations-and-countries-a1b55a08132e}).} The scale and speed of productionisation still far outstrips our understanding of NLP systems. 

In Part~\ref{part:measurement} I first examined the dominant method of bias evaluation in language models, based on embedding geometry and cosine similarity, and found it to have poor predictive validity. I advocated for measuring bias downstream. 
I next found an alternative upstream measure in information theoretic probing of demographics. This measure shows though only \textit{potential} for social bias downstream, a potential that may or may not be realised depending on classifier training. This both enables better understanding, but also reinforces that the full NLP system is still needed to accurately measure fairness.

In Part~\ref{part:crosslingual}, I examined monolingual and cross-lingual transfer in the setting of sentiment classification, in five languages, and found that both settings changed fairness outcomes. Monolingual transfer generally improved fairness despite that this introduces reams of foul pretraining data scraped from the depths of Common Crawl. I attributed this to increased stability in model decisions from the additional data: where stability is defined as not just fewer errors under the counterfactual, but smaller magnitude ones. 

Multilingual transfer, however, often worsened fairness outcomes, despite using more data than monolingual transfer. This may not contradict the previous result, as there are less parameters per language, such that even with regard to performance multilingual transfer models are not more stable than monolingual transfer models. The reasons for this difference are left to future work. 

In Part~\ref{part:generation}, I examined dense retrieval models, using the tools and research questions accumulated in Parts~\ref{part:measurement} and \ref{part:crosslingual}. I found that random initialisation and random data shuffle play a much larger role than previously thought, and that both performance and fairness were quite sensitive to them. This challenges the standard practice of using only one model with one initialisation and data shuffle for research. Using only one model was then common and is now ubiquitous in the age of LLMs, where training one model takes over 400,000 kWh \citep{luccioni_bloom_carbon}, or the same amount of energy to make 14 million cups of tea, approximately the same amount that is drunk in Scotland daily.\footnote{It takes about 336000 joules to raise 1 liter (4 cups of tea) of water to a boil, which is equivalent to 0.116 kWh assuming an electric kettle at 80\% efficiency. This makes an LLM equivalent to approximately 13,714,286 cups of tea. The UK drinks on average 3 cups of tea per day \citep{tea_stats}, and the population of Scotland is 5.4 million today.} 

In this work I also found a case where information theoretic probing was \textit{not} predictive of gender bias in retrievers, because the bias was caused by other factors beyond the model representation itself. 

\textbf{The combined work in this thesis repeatedly shows how little it makes sense to make choices or come to conclusions about fairness without understanding and simulating the entire system. }

\section{Questions}
Each individual work raised as many new questions as it answered.
%, which have not yet been answered by other analytical work. 
These questions are significant in both size and importance and would be valuable extensions to the understanding of the field, even in the era of LLMs. 

For both works in Part~\ref{part:measurement} the question remains: what about generative models?  What is the relationship between upstream language model metrics and downstream bias in generation, as opposed to in classification? The objective in generation is more similar to that of pre-training, so there is a chance that there is a more predictable correspondence between the two stages. Chapter~\ref{chapter:gender_bias_probing} additionally showed that whether bias was realised was a property of both the language model and the classifier, so what if there's no classifier? 
Future work could use some of the very recent progress in measuring biases in generation and measuring representational biases (discussed in \S\ref{sec:measuring_fairness}) to answer these questions. 
%But to answer that question, there has to be a way to measure bias in generation, which the field still needs more work on, 

In this section (Part~\ref{part:measurement}) for classification, I used \textbf{observational} studies to determine \textbf{allocational bias} by measuring whether the error rates were equal across different demographic populations (male group vs. female group, etc). In Part~\ref{part:crosslingual} I used \textbf{interventional studies} that measured whether a change in demographic variable changed predictions, when the predictions should be equal. Both of these measurements require a notion of \textbf{equality} --- which is very easy with a discrete label space or ordinal values. The same type of study could and should be done for generative models: instead of classifying resumes, a model could write summaries of resumes with a recommendation to proceed or not, which is then read by a human.\footnote{This is what is happening in practice with generative AI now, as I will discuss in the next section about my experience in Industry.} We can make the same invariance assertion as we made in classification: if we change the gender or race on the resume, the summary should not materially change. But what is a \textbf{material change}? In principle, it is a change that is large enough to cause the summary to be less accurate. Or to be equivalently accurate but to affect a human's opinion positively or negatively. Both would be good operationalisations for different contexts. How do we measure this? 

There is scarce previous work on counterfactuals in generative systems, but it raises just as many questions. \citet{vig_causal} consider social bias in generation to be the relative probability of professions like \textit{doctor} and \textit{nurse} under a counterfactual where male and female pronouns are swapped. How would that be extended to different grammatical systems, like Turkish, which has no gendered pronouns? What about different demographic biases that aren't encoded the way gender is? There is far less research about other demographic biases in this kind of setting. The comparatively thin coverage has been noted for race \citep{field-etal-2021-survey}, and there is vanishingly little on other demographic biases (with some exceptions of different coverage, such as \citet{hutchinson-etal-2020-social}, and of very broad coverage, such as \citet{esiobu-etal-2023-robbie}). But though they are less well represented in NLP, they are no less important from the viewpoint of ethics or of law. 
%Even with those worked out, this measure is so zoomed in that, while it matters, it's hard to see what scope of bias problems it can and cannot represent. 
%BELOW IS REDUNDANT
%\citep{sheng-etal-2019} use \textit{regard} of the demographic that a generation is about, which is generalisable as a notion, but not in practice when operationalising: they train one classifier, which is available in English only, for three binaries (male/female, white/black, and gay/straight) and has never been updated. 

% \citet{BOLD}, which is used as the standard of bias measurement in the LLM technical reports of today (e.g. Llama and Mixtral \citep{llama2,jiang2024mixtral}) create a dataset where they use gender term frequency and regard, in combination with sentiment and toxicity, to try to overcome the issues of each. 
% %Plenty of harmful stereotypes carry positive sentiment [example] and lots of social bias doesn't manifest via differences in toxicity [example. 
% More essential measurement groundwork has to be done for generation before we can begin to determine whether these metrics are predictable from upstream metrics.

The second work in Part~\ref{part:measurement}, Chapter~\ref{chapter:gender_bias_probing}, raises the question: what about beyond gender? Most work in this thesis by design looks at bias beyond gender (\ref{chapter:intrinsic_bias_metrics}, \ref{chapter:multilingual_sentiment_analysis}, and \ref{chapter:multilingual_sentiment_analysis_pt2} all include some notion of race or country of origin) but this one,  which proposes a new metric, looks only at gender (partly for lack of suitable datasets beyond it). But gender is encoded very differently in language than other demographic features, so it could reasonably have a different way it operates in model representations and social bias. In English, which weakly marks gender, and other languages with stronger gender agreement, gender information is necessary for correct grammar. A model will need to represent gender well for correct language reconstruction of any text from a noising objective, which is how Transformer models are currently trained  \citep{liu2019roberta, lewis-etal-2020-bart, vaswani}. But race and country of origin are not as strong signals. It is not easy to determine these save from specific words like names,and even then the signal is not as clear as with pronouns, as names do not \textit{just} encode race but also class, gender, time period, etc. How does this difference in encoded information affect the relationship between language models and downstream bias?

In Part~\ref{part:crosslingual}, Chapter~\ref{chapter:multilingual_sentiment_analysis}, we found that there was less bias in aggregate in monolingual transfer, and more reasonable patterns of bias, evidenced by less dramatic changes in sentiment score under the counterfactual. But what about tracing individual examples through from pre-training? Could we track a specific negative stereotype in pre-training and see if it affects decisions later? Extending to the work in Chapter~\ref{chapter:multilingual_sentiment_analysis_pt2}, could we extend tracing individual biases into multiple languages?\footnote{One work has recently come out that also shows `stereotype leakage' across languages \citep{cao2023multilingual}, which also helps form a foundation for this new question.} Almost all bias research is done on aggregate information, and we extended our focus to be on patterns of bias, but we stopped short of doing fine-grained analysis, which would be valuable. 

We've spent this thesis tracing how fairness persists and travels through a system at a macro level, but we could extend this to a micro level. Such research would not even be bias specific; for there isn't concrete knowledge yet of how \textit{any} information travels between different training stages of models (of which there are increasingly many in the age of LLMs).\footnote{There are some methods like this that are starting to do this, like influence functions \citep{grosse2023studying} and some methods that try to do this with Natural Language explanations, which is nonsense and does not work, as \citet{huang-etal-2023-rigorously} found `no evidence for causal efficacy' of them.}

In Background Section~\ref{sec:fairness_as_other_fields}, I introduced the notion of fairness as a generalisation error vs. as a learnt dataset artifact---whether an artifact from spurious correlation or a historical bias. Investigation into this difference in causes could help enlighten why racial bias can increase with cross-lingual transfer. Is it really compounding biases (stereotypes) or could it be a generalisation error? %One of these types of bias could have much less certainty than the other. 
Can an investigation into model uncertainty help illuminate which of these cases causes the effects we have observed?

Part~\ref{part:generation} shares the question of `biases beyond gender' from Chapter \ref{chapter:gender_bias_probing}, as it is also solely gender focused. It also raises questions are general to our understanding of NLP systems as a whole, but have particular importance to fairness. Why is random seed initialisation so important for bias and for generalisation? Why is it possible for a couple of seeds to \textit{just not work at all}, never mind fairness? Some of the anistropy of the representations from earlier training stages seems potentially predictive of later behaviour. Can we understand this well enough to utilise it? If so we could potentially be able to actively encourage model training that is less prone to shortcutting. 

\section{Present Day}
Now I want to bring this into current industry practice and zeitgeist. I do this partly because I've spent the better part of the last year working full-time on fairness at an LLM company (Cohere). And partly because, in reviewing my PhD work, I don't want to ignore the sea change in NLP research that's taken place over the past year and a hald. I am not someone for whom `\textit{scaling is a way of life}'\footnote{This light shade given by Tatsunori Hashimoto when questioned about it at GenBench at EMNLP 2023.}, but it would be disingenuous, in a field intended to improve people's lives, to not speak about how my work relates to current research, current discourse, and current industry practice. This thesis was initially inspired by a sea change that I saw happening six years ago, after all. 

This work was all done on models three orders of magnitude smaller than the ones that I deal with in my work today. 

This does not matter as much as it might seem. No conclusions in this thesis were model specific. If some architecture arises to replace neural embeddings, LSTMs, and Transformers which bears no genetic link to them, then they \textit{may} no longer hold. But until then, the differences between the models I use at work today and the models in this thesis are: 1) scale 2) a veneer of preference tuning (RLHF, DPO, etc) \citep{rafailov2024direct} 3) instruction tuning \citep{ouyang2022training} and 4) more training data that is explicitly in the domain of math, logic, and code than we used to include in general NLP models.\footnote{Though some amount of math and code will be present in CommonCrawl, which does drive the models in this thesis.} None of these differences affect my conclusions. 

In my first rebuttal to ACL reviewers when the work in Chapter~\ref{chapter:intrinsic_bias_metrics} was under review, one of the reviewers asked the common reviewer question `But have you tried this on {\tt Newest Model Architecture}' (which in this case, was BERT). Adam gave me the advice to turn that into a the question: `Is there any reason to expect that {\tt Newest Model Architecture} would behave differently? Otherwise, they're just saying it is magic'. To answer this question broadly for LLMs: there is no reason to believe any of the four recent innovations change the things we discovered about fairness in this thesis.

Here are some examples of this being proven. 

The replication and extension of my work in Chapter~\ref{chapter:intrinsic_bias_metrics} by \citet{cao-etal-2022-intrinsic} did use BERT, and 18 other transformer architectures of varying sizes, and came to the same conclusions. We've seen the same bias amplification affects in LLMs at scale \citep{bianchi-2023} that we saw in small models in \citet{zhao-etal-2017-men}. 

Current research shows that RLHF and the family of preference tuning algorithms predominantly affect \textit{style and structure} of generation, rather than content \citep{min-etal-2022-rethinking,  lin2023unlocking}. More research shows it can be trivially changed with a few dozen fine-tuning examples \citep{qi2023finetuning} and that it quickly `wears off' over conversation turns \citep{llama2}. All information is learnt at earlier stages, predominantly pretraining \citep{zhou2023lima}. So from this we conclude that preference tuning will not affect our conclusions. 

There is no research I have seen that enables inferences on the effect of instruction tuning or the inclusion of more code and math in data, but there's no reason a priori to think they would change fairness behaviour.  

There is one salient change that will matter. Language models are trained to compress and then reconstruct the data they were trained on, and this lossy compression has become less lossy as an effect of scaling. That is: LLMs memorise more individual training samples \citep{karamolegkou-etal-2023-copyright}. This could change fairness outcomes, though will it help or will it harm? This depends somewhat on whether the source of the unfairness is a dataset artifact or a generalisation error (\S \ref{sec:fairness_as_other_fields}). On the one hand, overall increased memorisation is likely to exacerbate the learning of artifacts. On the other hand, we don't yet understand how scaling affects generalisation, as it is too difficult to test in the current era of closed language models and unknown pretraining and fine-tuning data. 

Regardless of this, scaling won't affect the measurements or mechanisms of bias transfer. But these potential interactions of scaling do lend weight to the need for more work on disentangling sources of bias and looking at the effects of increased memorisation from overparameterisation. To date almost all work on memorisation has been from the viewpoint of copyright \citep{karamolegkou-etal-2023-copyright}, security and privacy \citep{smith2023identifying, hartmann2023sok}, or rarely, model quality (where memorisation is at odds with generalisation) \citep{tanzer-etal-2022-memorisation}. The NLP community should also look at it from the viewpoint of fairness.
 
When I started this thesis I focused on validating metrics, not because of a dedication to evaluation; I had grand plans for applying my ideas to cross-lingual bias mitigation. But I'd seen unvalidated assumptions in the standard metrics of the field, and it made me unwilling to use those metrics in my own work. I didn't want to stake my PhD research on a metric that I didn't trust, and find out 1.5 years in that my intuition not to trust it had been correct. But now that I work on a deployed product, I spend at least half my time on evaluation. Because good evaluation was \textit{always very hard} and the rise of generative AI has only made it harder. And I can only throw darts at a wall (a perhaps unfair caricature of LLM training) if I know when they've hit something useful, and \textit{that's} the hard part, not the dart throwing.

There is some irony in how Chapter~\ref{chapter:intrinsic_bias_metrics}, my first fairness work, was the seminal work showing that you cannot do upstream social bias mitigation, and then I took a job where I am supposed to do just that. In practice, I need to try, since education about NLP systems is not yet good enough, and the deployers of language models do not yet have the knowledge and resources to do bias mitigation themselves. So I use the tools and discoveries that I made over the course of this thesis to evaluate my models, and measure wide bounds for what types of bias \textit{could} occur in different reasonable settings, and then make this information public, so that deployers know, can work around it, and maybe do something about it.

But this is still not satisfying enough. I do not think we will ever get to a point in which we rely on one single large pretrained model for thousands of use cases and can predict bias effects downstream for anything but the most common ones. All of this research has progressively taught me that I need to consider the entire NLP system in my measurements for bias: the pretraining, the fine-tuning, the task, the inputs, the corpus that a model can query. The limit case of this it that I need to consider the user interface, the users themselves, the societal power structures within which the NLP system is embedded. And I do think, at some stage, these need to be part of NLP experimental conditions. We cannot consider the harmful effects of QA systems providing false information in absence of how it is displayed in a UI, and how much that UI encourages trust or overreliance \citep{bucinca_2021}. Bias research cannot consider stereotypes in absence of the power structures that make them harmful \citep{blodgett-etal-2021-stereotyping}. No more can most NLP systems be considered without these things, which all together make it increasingly complex to predict all of these things at an upstream stage. 

But we can get to a point where we understand better the effect of the choices we've made in the life-cycle of an NLP system. Which ones tend to make things worse, which better, and why. With that, we can better predict potential bias in new systems, and then allocate evaluations and mitigation methods accordingly. But first, we need to understand our systems as a whole.
%So that we can take a set of facts about a model, and then generalise from it. %like humans do





% NOTES:
% The role of fairness research is the understanding that industry doesn't have time to do but that can still have hope of being applied in practice of having bearing on a real world situation -- this is a characteristic of the field since it is necessarily grounded in real humans and their lives.


%We are building out the picture of the full ecosystem of what can matter over the course of the thesis




% WHAT else can I bring in about science about goodharts law about things I've learnt about doing this all in practice?


%%% Outline what I want to cover in the discussion

\bibliographystyle{apalike}
\bibliography{anthology,anthology_p2,bibl}

\appendix

\end{document}
